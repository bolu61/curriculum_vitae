\documentclass[10pt,letterpaper]{article}
\usepackage[
    ignoreheadfoot,
    top=1cm,
    bottom=1cm,
    left=1cm,
    right=1cm,
    footskip=1cm
]{geometry}
\usepackage{titlesec}
\usepackage{enumitem}
\usepackage[hidelinks]{hyperref}
\usepackage{xcolor}

\newcommand{\datedsection}[2]{\section[#1]{#1 \hfill #2}}
\newcommand{\datedsubsection}[2]{\subsection[#1]{#1 \hfill #2}}
\newcommand{\name}[1]{\centerline{\Huge{#1}}}
\newcommand{\contact}[1]{\centerline{#1}}
\newcommand{\sep}{{\large\textperiodcentered}}

\definecolor{basecolor}{cmyk}{0,0,0,0.8}
\definecolor{emphcolor}{cmyk}{0,0,0,1.0}
\color{basecolor}
\renewcommand{\emph}[1]{{\color{emphcolor}\bfseries#1}}

\titleformat{\section}
    {\Large\scshape\raggedright}
    {}
    {0em}
    {}
    [\titlerule]
    
\titleformat{\subsection}
    {\large\raggedright}
    {}
    {0em}
    {}
    
\titlespacing{\section}{0pt}{2pt}{2pt}    

\titlespacing{\subsection}{0pt}{0pt}{0pt}

\setlist{nolistsep,noitemsep,leftmargin=1.5em}
    
\begin{document}

\name{Jianchen Zhao}
\contact{\href{mailto:jianchen.zhao@uwaterloo.ca}{jianchen.zhao@uwaterloo.ca} \sep{} \href{tel:+15147265029}{+1 (524) 726-5029} \sep{} \href{https://github.com/bolu61}{github.com/bolu61} \sep{} \href{https://orcid.org/0000-0002-6362-0106}{orcid.org/0000-0002-6362-0106}}

\section{Education}

\datedsubsection{\emph{McGill University}}{2017 to 2020}
\noindent B.Sc. in Honours Computer Science with a Minor in Mathematics \emph{3.6/4 GPA}

\datedsubsection{\emph{University of Waterloo}}{2021 to present}
\noindent \emph{Ph.D.} in Electrical and Computer Engineering \emph{3.9/4 GPA}

\section{Publications}

\subsection{Practical Preprocessing of Logs at Scale}
\noindent ICSE 2025 Doctoral Symposium.
\begin{itemize}
    \item Doctoral Thesis work in progress, with a proposal accepted at ICSE 2025 Doctoral Symposium.
    \item I develop new methods to preprocess large scale and distributed
    software logs for storage and analysis, such as \emph{entropy} based
    log-parser and console log segmentation using \emph{Transformers}.
    \item Uses a combination of the latest technique in \emph{AI}, including
    \emph{Machine Learning} (interleaved hidden Markov models) and \emph{Deep
    Learning} (recurrent neural networks, transformers, LLMs) implemented using
    \emph{Python} \emph{JAX} and \emph{PyTorch}.
\end{itemize}

\datedsubsection{Studying and Complementing the Use of Identifiers in Logs}{2023}
\noindent J Zhao, Y Tang, S Sunil, W Shang. IEEE International Conference on Software Analysis, Evolution and Reengineering (SANER), Taipa, Macao,
2023 pp. 97-107.
\begin{itemize}
    \item Exploration of the relationship of logging statements in complex
    software systems using \emph{static program analysis} and data flow analysis.
    \item Implemented \emph{SCNA} to compute dominator trees of logging statement \emph{CFG} nodes.
    \item Performed \emph{empirical study} and case study across many open-source
    projects to determine the relationship of logging statements.
\end{itemize}

\section{Experience}

\datedsubsection{Morgan Stanley (Technology Analyst)}{2018}
\begin{itemize}
    \item Implemented a \emph{data warehousing} system for managing internal messaging and \emph{VoIP} QA data. 
    \item Designed a \emph{full stack} application to generate reports into
    dashboards, using \emph{Python} as a backend.
    \item Worked on a based \emph{AI} chat bot framework using \emph{Python} for
    internal message channels.
\end{itemize}

\datedsubsection{Wind River (Researcher)}{2023 to 2024}
\begin{itemize}
    \item Developed automotive \emph{data driven testing} framework, reducing testing cost by 80\%.
    \item Used \emph{Python}
    \item Used novel approach by leveraging system requirement graph and
    \emph{NSGA-III} genetic search to find optimal test set.
    \item Engineered features from system requirement graph and trained dozens of cross validated
    \emph{Machine Learning Models} for ranking testing priorities.
\end{itemize}

\section{Projects}
\subsection{Prefixspan github.com/bolu61/prefixspan}
\begin{itemize}
    \item Implemented the classic \verb|prefixspan| algorithm using \emph{C++20}
    with \emph{Python} bindings with a c++20 concept based generic interface.
\end{itemize}
\subsection{Deepspan github.com/bolu61/deepspan}
\begin{itemize}
    \item Designed a pattern matching algorithm to perform sequencing of interleaved events.
    \item Uses both traditionnal \emph{machine learning models} such as a
    mixture of \emph{hidden Markov models} and \emph{deep learning} models based
    on \emph{LSTM}, \emph{Transformers} and \emph{GRU}s.
\end{itemize}

\datedsubsection{HackTheNorth \href{https://devpost.com/software/snapevnt}{devpost.com/software/snapevnt}}{2018}
\begin{itemize}
    \item Designed and implemented a realtime chat application with \emph{AI} powered
    event management tools using Android SDK.
\end{itemize}
\datedsubsection{Hackatown \href{https://devpost.com/software/sureviews}{devpost.com/software/sureviews}}{2019}
\begin{itemize}
    \item Designed and implemented an \emph{AI} restaurant review summarization
    tool utilizing GCP \emph{NLP} API.
\end{itemize}
\datedsubsection{McHacks \href{https://devpost.com/software/utrition}{devpost.com/software/utrition}}{2019}
\begin{itemize}
    \item Designed and implemented an \emph{AI} food Ingredient analyzer that
    uses \emph{image-to-text} models to analyze ingredients list on food
    packaging.
\end{itemize}

\section{Skills}
\vspace{1pt}
\noindent \emph{C++20}, \emph{Python}, \emph{JAX}, \emph{PyTorch}, rust, C, AI, Machine Learning, LLM, RAG, SQL, Linux, Git, Empirical Research, Full Stack Development, Static Analysis

\end{document}
